\documentclass[11pt]{article}
\usepackage[margin=0.8in]{geometry}
\usepackage{graphicx} % Required for inserting images
\setlength{\parindent}{0pt}
\usepackage{float}
\usepackage{caption}

\title{{\fontsize{21pt}{18pt}\selectfont COP290 Assignment - 1 Subtask - 2} \\ Trading Simulator and Analyzer}
\author{Yash Bansal \\ 2022CS51133 \and Shivam Sawarn \\ 2022CS11075}
\date{}
\begin{document}

\maketitle

\section{Introduction}

\begin{itemize}
    \item This is a trading website \textbf{INFTY} that helps users view information about different NSE stocks, with a major emphasis on NIFTY 100 stocks.
    \item For good user experience, different kinds of graphs are available, which helps the user visualize the data and thus helps to get more useful info on the stocks.
    \item Different features like OTP Registration, password reset, and filters are added to a more user-friendly website.
\end{itemize}

\section{Features and implementation}
Our website has the following major features:-

\subsection{Login page:-}
\begin{itemize}
    \item This page is to log in to the dashboard of the user if the given credentials are correct.
    \item If credentials are not correct, then it redirects the user again to the login page, with the error message to the user to enter the correct login credentials.
\end{itemize}

\subsection{Register page:-}
\begin{itemize}
    \item To register, the user needs to enter the username, phone number, and email ID as user details, which need to be unique for all the users. The registration page will not get submitted if the details are not valid or they already exist for any other user.
    \item For additional security, the confirmed password is also added so that the user can be sure about the password he used.
    \item After submitting the registration page, an OTP is sent to the email ID of the user, and the user is redirected to the OTP verification page. If the entered OTP is correct, registration is successful, and the user can then log in to his dashboard.
    \item A resend OTP option is also added so that the user can again receive OTP in case of not receiving the OTP.
\end{itemize}

\subsection{Password reset:-}
\begin{itemize}
    \item If the user forgets his password, then he can easily reset his password by receiving an OTP in his registered email id.
\end{itemize}

\subsection{User Database:-}
\begin{itemize}
    \item For the user database, we used the SQLAlchemy library, which generates a highly efficient SQL database, supporting efficient search and extract data implementations.
\end{itemize}

\subsection{Dashboard:-}
\begin{itemize}
    \item After login into his dashboard, the user can see the graphs of live data of different stocks of NSE's NITFY 100 stocks.
    \item Also, two or more graphs can be plotted on a single plot to ease the comparison between different stocks.
    \item The plots can be formed for different stocks, multiple days, and opening and closing prices.
    \item In the top right corner, the user can see his profile, which shows his username and email id. From there, the user can also toggle between light and dark mode.
    \item From the logout option, the user can safely log out from the website, thereby disabling him to access his data without logging in again.
    \item A search bar is also available, which provides an efficient search for stocks, and the user can go to the website of different stocks by clicking directly on the tab of the stocks.
    \item The end of the webpage contains a filter option, which can filter the stock based of the given parameter's maximum and minimum values.
\end{itemize}

\subsection{Webpage of each stock:-}
\begin{itemize}
    \item The webpage of the stocks contains the different live parameters of the stocks, namely open price, close price, high, low and other relevant features, which could be used by the user.
    \item Other important historical statistics are also available like P/E ratio, market debt, totak debt and total revenue.
    \item The page contains a good interative graph, which contains the data of the day in a good interactive format. The user can select from the bottom panel of the graph to select the time between which he wants the data.
\end{itemize}

\end{document}
